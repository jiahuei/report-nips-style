\documentclass{article}

% To use squared citation format:
%\PassOptionsToPackage{square, numbers}{natbib}
\usepackage[square, numbers]{natbib}
\bibliographystyle{elsarticle-harv}

\usepackage[final,titlepage]{report_2019_JH}


% Packages imported in NIPS tex file
\usepackage[utf8]{inputenc}         % allow utf-8 input
\usepackage[T1]{fontenc}            % use 8-bit T1 fonts
\usepackage{hyperref}               % hyperlinks
\usepackage{url}                    % simple URL typesetting
\usepackage{booktabs}               % professional-quality tables
\usepackage{amsfonts}               % blackboard math symbols
\usepackage{nicefrac}               % compact symbols for 1/2, etc.
\usepackage{microtype}              % microtypography


\newcommand{\universityLogo}{images/logo1.png}              % Put university logo path here
\newcommand{\university}{My University / Faculty Name}      % Put university name here

\title{My Second Report}
\author{Your Name \\ \textit{WHA 160016} \\ \AND
        Co-author \\ \textit{Supervisor} \\ \AND
        Co-author \\ \textit{Co-supervisor}}

\begin{document}


\maketitle

\tableofcontents
\listoffigures
\listoftables
\clearpage

%-------------------------------------------------------------------------
%-------------------------------------------------------------------------
%-------------------------------------------------------------------------


\section{Introduction}
\label{sec: Introduction}

Nothing to introduce here. Just a collection of maths.



%-------------------------------------------------------------------------
%-------------------------------------------------------------------------
%-------------------------------------------------------------------------


\section{Products}
\label{sec: Products}


%-------------------------------------------------------------------------
\subsection[Outer product]{Outer product / Tensor product}
\label{subsec: Outer product}

From \cite{wiki2019outerProd}.

For vectors or column matrices, it is equivalent to a matrix multiplication $x y^{T}$:

\begin{equation}
    \mathbf{u} \otimes \mathbf{v} = \mathbf{u} \mathbf{v}^{T} = \left[ \begin{array}{cccc}{u_{1} v_{1}} & {u_{1} v_{2}} & {\ldots} & {u_{1} v_{n}} \\ {u_{2} v_{1}} & {u_{2} v_{2}} & {\dots} & {u_{2} v_{n}} \\ {\vdots} & {\vdots} & {\ddots} & {\vdots} \\ {u_{m} v_{1}} & {u_{m} v_{2}} & {\ldots} & {u_{m} v_{n}}\end{array}\right]
\end{equation}
\noindent
where $u, v \in\mathbb{R}\:^{n \times 1}$

For higher order tensors:

\begin{equation}
    \mathbf{T} = \mathbf{a} \otimes \mathbf{b} \otimes \mathbf{d}, \quad T_{i j k}=a_{i} b_{j} d_{k}
\end{equation}

For example, if $\mathbf{A}$ is of order $3$ with dimensions $(3, 5, 7)$ and $\mathbf{B}$ is of order $2$ with dimensions $(10, 100)$, their outer product $\mathbf{C}$ is of order $5$ with dimensions $(3, 5, 7, 10, 100)$.

Tensor product is a generalisation of outer product \cite{wiki2019tensorProd}. The outer product of two vectors $\mathbf{u}$ and $\mathbf{v}$ is their tensor product $\mathbf{u} \otimes \mathbf{v}$ \cite{wiki2019outerProd}.


%-------------------------------------------------------------------------
\subsubsection{A sub-subsection}
\label{subsubsec: A sub-subsection}

Lorem ipsum dolor sit amet, consectetur adipiscing elit. Maecenas pulvinar leo id felis pulvinar aliquet. Phasellus placerat dignissim odio aliquam hendrerit. Nullam vel justo eget ex feugiat congue. Pellentesque sagittis congue ipsum. Sed dapibus, nibh eu blandit efficitur, arcu nisi suscipit risus, dignissim iaculis mi dui ut massa. Phasellus id sem tempus, congue nunc id, placerat quam. Sed gravida, erat vel finibus egestas, leo augue pretium turpis, at commodo enim augue vitae ipsum.


%-------------------------------------------------------------------------
\paragraph{A sub-sub-subsection}
\label{paragraph: A sub-sub-subsection}

\begin{figure}[h]
	\begin{center}             % figure uses center environment
        \includegraphics[width=0.2\linewidth]{images/logo2.png}
    \end{center}
    \caption[A logo.]{A logo. Please don't sue me.}
    \label{fig: Logo}          % label is always after caption
\end{figure}


Etiam semper felis vitae purus dapibus, sit amet luctus massa tempor. Donec neque elit, vehicula eget facilisis vitae, rhoncus vel arcu. Nunc ultrices placerat felis eu vulputate. Praesent nunc arcu, rhoncus id accumsan vitae, tempus vitae velit. Maecenas vel nisl sit amet elit pulvinar hendrerit at at metus. Maecenas aliquet ullamcorper facilisis. Aliquam erat volutpat. Vestibulum quis diam nunc. Nunc sed sollicitudin orci. Nunc ut sem urna. Nunc blandit diam orci. Curabitur non pharetra diam, nec imperdiet urna. Fusce luctus nisl non nisi congue egestas. Nunc tincidunt commodo diam, et sagittis neque luctus malesuada. Curabitur auctor felis eu congue venenatis.


%-------------------------------------------------------------------------
\paragraph{Another sub-sub-subsection}
\label{paragraph: Another sub-sub-subsection}

\begin{table}[h]
    \caption[A table.]{A table from Wikibooks.}
    \label{table: Table}          % label is always after caption
    \begin{center}
        \begin{tabular}{ l | l  l  p{5cm} }
        \toprule
        Day & Min Temp & Max Temp & Summary \\ \midrule
        Monday & 11C & 22C & A clear day with lots of sunshine.  
        However, the strong breeze will bring down the temperatures. \\ \midrule
        Tuesday & 9C & 19C & Cloudy with rain, across many northern regions. Clear spells 
        across most of Scotland and Northern Ireland, 
        but rain reaching the far northwest. \\ \midrule
        Wednesday & 10C & 21C & Rain will still linger for the morning. 
        Conditions will improve by early afternoon and continue 
        throughout the evening. \\
        \bottomrule
        \end{tabular}
    \end{center}
\end{table}


Nam dictum, mi non ornare semper, nunc tellus vehicula risus, quis blandit tortor velit et ex. Morbi viverra sit amet magna quis convallis. Etiam et neque dapibus, lobortis neque vestibulum, placerat mi. Vivamus ut ipsum vel enim blandit pellentesque. Integer nec nisl tellus. Donec aliquam accumsan ipsum, nec laoreet dui auctor id. Nullam rutrum augue neque, quis vestibulum tellus dapibus dictum. Vestibulum placerat metus et lectus sagittis venenatis.


%-------------------------------------------------------------------------
\subparagraph{A sub-sub-sub-subsection}
\label{subparagraph: A sub-sub-sub-subsection}

Nunc efficitur, mi id efficitur tempor, nulla diam facilisis risus, sit amet commodo ex arcu eu sem. Aliquam metus nisi, rhoncus in dictum at, posuere ut metus. Nunc tristique nisl eu velit sagittis, ut ornare odio porta. Sed mi nisl, tincidunt nec vehicula eget, accumsan quis urna. In hac habitasse platea dictumst. Praesent commodo sollicitudin pulvinar. Nullam pulvinar aliquam velit, ut fermentum augue convallis eu. Mauris interdum feugiat mollis.


%-------------------------------------------------------------------------
\subsubsection{Another sub-subsection}
\label{subsubsec: Another sub-subsection}

Vivamus ut nisl sagittis, ullamcorper felis eget, dictum arcu. Sed placerat aliquam tortor, sit amet sodales eros consequat ac. Donec imperdiet, dui eget rutrum sollicitudin, nisi tellus cursus felis, non fringilla sem nunc porta magna. Quisque enim arcu, porttitor et dignissim at, porttitor sed leo.



%-------------------------------------------------------------------------
%-------------------------------------------------------------------------
%-------------------------------------------------------------------------


\clearpage


%\renewcommand\refname{Bibliography}
\bibliography{ref}
\end{document}
